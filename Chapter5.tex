\chapter{Conclusion}\label{ch:con}


To conclude this project, a few closing remarks will be done to give a general scope of the project, its limitations and possible future work that could be done to extend the results found in this thesis. First, let us remember the  aim of this thesis, which was ''to study the effects of nonlinear noise in a coherent optical communication system and suggest a \textit{nonparameter digital signal processing detection scheme} enabled by machine learning''. The nonparameter scheme suggested used \textit{support vector machine} and \textit{random forest} as a statistical tool to classify symbols detected in a coherent optical communication link. Nonlinear effects were simulated for two different coherent optical communication links, a long-haul link and a \textit{wave length division multiplexing} system.

For the long-haul fiber link simulated, the optimal launch power was determined to be 1~mW. It was concluded that for the optimal launch power, the nonparameter digital processing scheme could preform nonlinear phase noise compensation comparable to analitycal scheme that has all the information of the links architecture. It was determined that the nonparameter  scheme could potentially improve the transmission length between 2 to 9 times further. The main observation for this simulation was that for nonlinear phase noise, knowledge of the system can be replaced by training a machine learning algorithm to learn the impairments of the link. A fundamental limitation of the simulation carried out was the amount of transmitted symbols. Given that when the \textit{symbol error rate} is low (when few mistakes are made) more symbols are need to accurately measure the error rate (more symbols are need to make a mistake). Given the limitation in simulation run time, it was not possible to find a precises optimal transmission length for the nonparameter scheme. All conclusion had to be drawn from the \textit{optimal prediction region} of the processing scheme.

In the multiplexed system it was determined that no clear benefit could be determined in implementing machine learning to mitigate interchannel noise. Suggesting that \textit{cross phase modulation}, the dominant noise source in the link is a time dynamic effect and can not be resolved with the digital signal processing scheme suggested in this report. However, the joint implementation with other technologies could help improve the reliability of the system. This simulation was also limited by the amount of symbols simulated. If more symbols could be simulated it could determined if there is clear benefit in implementing the proposed scheme. 

The research carried out in this project could be extended in a couple of different ways. To get a clear measurement of the optimal transmission length a simulation with $2^{17}$ symbols could carried out. This was not possible in this project because of increase in computing time throughout the whole implementation. An other relevant aspect that was not pursued was investigating the effect of using different constellation diagrams, in this project only a square 16QAM constellation was implemented. Depending on the constellation the amount of acquired nonlinear phase noise changes. One last measurement that could be preformed is the effect on changing the fiber arrangement in the multiplexed link. Depending on the dispersion management implemented the interaction between the carrier signal optical power and the fibers Kerr effect could be altered. Given a clever fiber arrangement a system with less nonlinear noise could be constructed.

As a final remark, in this project it was observed that some of the nonlinear noise present in a fiber link could be compensated without needing any information on the systems physical state. Suggesting that machine learning as a statistical tool could help reduce the deployment complexity of a network. It can also be mentioned that the proposed processing scheme can learn form the real impairments of the system which is desired. The limitation of the processing scheme lay in its inability in resolving time dynamic noise sources. Personally, the lesson I take away from this project is simple, in some cases knowledge can be substituted by training. Which is not only true for nonlinear phase noise compensation in coherent long-haul fiber system, but for life in general.    