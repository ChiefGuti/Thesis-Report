%%%%%%%%%%%%%%%%%%%%%%%%%%%%%%%%%%%%%%%%%%%%%%%%%%%%%%%%%%%%%%%%%%%%%%%%%%
%   This is frontpage.tex file needed for the dmathesis.cls file.  You   %
%  have to  put this file in the same directory with your thesis files.  %
 % 
%                                                                        % 
%%%%%%%%%%%%%%%%%%%%%%%%%%%%%%%%%%%%%%%%%%%%%%%%%%%%%%%%%%%%%%%%%%%%%%%%%%%
%%%%%%%%%%%%%%%%%%%%%%%%%%%%%%%%%%%%%%%%%%%%%%%%%%%%%%%%%%%%%%%%%%%%%%%%%%%
%%%%%%%%%%%%%%%%           The title page           %%%%%%%%%%%%%%%%%%%%%%%  
%%%%%%%%%%%%%%%%%%%%%%%%%%%%%%%%%%%%%%%%%%%%%%%%%%%%%%%%%%%%%%%%%%%%%%%%%%%
\pagenumbering{roman}
%\pagenumbering{arabic}

\setcounter{page}{1}

\newpage


\thiswatermark{\centering \put(112,-550){\transparent{0.15}\includegraphics[scale=2]{LundUniversity_C_RGB.png}} }

\thispagestyle{empty}
\begin{center}
   \vspace*{1cm}
  {\large \textbf{Master's thesis in  }}

  \vspace*{0.5cm}
  {\LARGE \bf\noindent \textbf{Machine Learning Enabled Nonlinear Phase Noise Mitigation for a Coherent Optical Systems  }}

  \vspace*{1cm}
  {\large\bf Sergio Steven Guti\'errez Pulgar\'in}
  
  \vspace*{0.30cm}
	{\normalsize \bf Supervisor: Dr. Cord Arnold\hfill   Co-supervisor: Dr. Sergei Popov}
	
  \vfill

  {\large A thesis submitted in partial fulfillment of the requirements\\
  [1mm] for the degree of Master of Photonics}
  \vspace*{0.9cm}
  
  % Put your university logo here if you wish.
%   \begin{center}
   \includegraphics[width= 7cm]{LundUniversity_C_BLACK.png}
%   \end{center}

  {\large Department of Physics \\
          [-1mm] Atomic Physics Division\\
          [-1mm] Lund, Sweden\\
          [1mm]  May, 2020}
%          
%  \vspace*{0.30cm}
%	{\normalsize \bf Supervisor: Dr. Cord Arnold\hfill   Co-supervisor: Dr. Sergei Popov}
%	
%	\vspace*{0.10cm}	
%	
%		{\normalsize \bf \hfill Co-supervisor: Dr. Sergei Popov}


\end{center}

%%%%%%%%%%%%%%%%%%%%%%%%%%%%%%%%%%%%%%%%%%%%%%%%%%%%%%%%%%%%%%%%%%%%%%%%%%%
%%%%%%%%%%%%%%%% The dedication page, of you have one  %%%%%%%%%%%%%%%%%%%%  
%%%%%%%%%%%%%%%%%%%%%%%%%%%%%%%%%%%%%%%%%%%%%%%%%%%%%%%%%%%%%%%%%%%%%%%%%%%
\newpage
\thiswatermark{\centering \put(112,-550){\transparent{0.15}\includegraphics[scale=2]{LundUniversity_C_RGB.png}} }

\thispagestyle{empty}
\begin{center}
 \vspace*{2cm}
  \textit{\LARGE {Dedicated to}}\\ 
the endless source of knowledge that we all strive to be part of. Life has given me the chance to pursue my dreams, light has given me a way to understand my life. 



\end{center}


%%%%%%%%%%%%%%%%%%%%%%%%%%%%%%%%%%%%%%%%%%%%%%%%%%%%%%%%%%%%%%%%%%%%%%%%%%%
%%%%%%%%%%%%%%%%%%           The abstract page         %%%%%%%%%%%%%%%%%%%%  
%%%%%%%%%%%%%%%%%%%%%%%%%%%%%%%%%%%%%%%%%%%%%%%%%%%%%%%%%%%%%%%%%%%%%%%%%%%
\newpage
\thispagestyle{empty}
\addcontentsline{toc}{chapter}{\numberline{}Abstract}
\begin{center}
  \textbf{\large Machine Learning Enabled Nonlinear Phase Noise Mitigation for a Coherent Optical Systems}

  \vspace*{1cm}
  \textbf{\normalsize Sergio Steven Guti\'errez Pulgar\'in}

  \vspace*{0.5cm}
  {\normalsize Submitted for the degree of Master's in Photonics\\ May 2020}

  \vspace*{1cm}
  \textbf{\large Abstract}
\end{center}

\noindent In the current development of coherent optical communication systems, nonlinear noise is considered to be the ultimate bottleneck when extending the transmission length. In this report we suggest a nonparamter digital signal processing scheme to extend the transmission length of a fiber link. The processing scheme was enabled by machine learning,  implemented to compensate nonlinear noise in a communication system without needing any information about the physical state of the transmission line. It was shown that in the case of nonlinear phase noise in a long-haul fiber system, the proposed processing scheme could extend the transmission length of the fiber link. However, for interchannel noise a clear benefit could not be determined due to limitations in the simulation. It was concluded that for a long-haul fiber link, knowledge of the system could be replaces with learning through an optimal statistical algorithm. 


%%%%%%%%%%%%%%%%%%%%%%%%%%%%%%%%%%%%%%%%%%%%%%%%%%%%%%%%%%%%%%%%%%%%%%%%%%%
%%%%%%%%%%%%%%%%%%     The acknowledgements page         %%%%%%%%%%%%%%%%%%  
%%%%%%%%%%%%%%%%%%%%%%%%%%%%%%%%%%%%%%%%%%%%%%%%%%%%%%%%%%%%%%%%%%%%%%%%%%%
\chapter*{Acknowledgements}
\thiswatermark{\centering \put(0,-695){\includegraphics[width=3.2cm]{LundUniversity_C_RGB}} \put(260,-695){\transparent{1}\includegraphics[width=2.9cm]{kthlogo}\includegraphics[width=2.9cm]{riselogo}} }

I want to express my gratitude to my supervisor Cord Arnold for all his constant support throughout my whole master's. In the past two years Cord has given me all the tools to pursue the path I desired with my education at Lund. Special thanks to Cord for encouraging  me to contact the Faculty of Optical Communication at \textit{Kungliga Tekniska Högskolan} (KTH Royal Institute of Technology) for my thesis project.	  

The research carried out in this report was only possible thanks to the guidance and expertise of Sergei Popov. In the months I spend in Stockholm Sergei oversaw and helped me overcome difficulties in my project. I want to thank him for the opportunity to work at KTH University and for facilitating the assistance of the \textit{ Kista High-speed Transmission Lab} (Kista HST-Lab) at the \textit{Research Institutes of Sweden} (RISE).	 This project was not only a a great professional opportunity but it was the experience of a life time.

I would like to mention and express my appreciation to Aleksejs Udalcovs, Xiaodan Pang and Oskars Ozolins from the joint group KTH/RISE for their input throughout my thesis. With their expertise I was able to understand more in depth the objective and results of my project. Thanks to their input I was able to acquire a technical understanding of the simulations implemented.

In a personal note I want to thank Stefan Höst for listening to me through my whole project and giving me valuable insight from an Information Theory point of view.   

The physical and technical contribution of Kista High-speed Transmission Lab and KTH Royal Institute of Technology  is truly appreciated. Without their support  this project could not have reached its goal. To conclude I want to thank \textit{Lund University}, this will always be the start of a new chapter in my life, and I will always be grateful for all the experiences I have had in this journey. ~\\


 
Sergio Steven Gutiérrez





%%%%%%%%%%%%%%%%%%%%%%%%%%%%%%%%%%%%%%%%%%%%%%%%%%%%%%%%%%%%%%%%%%%%%%%%%%%
%%%%%%%%    tableofcontents, listoffigures and listoftables       %%%%%%%%%
%%%%%%%%        Command if you do not have  them                  %%%%%%%%%
%%%%%%%%%%%%%%%%%%%%%%%%%%%%%%%%%%%%%%%%%%%%%%%%%%%%%%%%%%%%%%%%%%%%%%%%%%%
\tableofcontents
\listoffigures
\lstlistoflistings
\clearpage


%%%%%%%%%%%%%%%%%%%%%%%%%%%%%%%%%%%%%%%%%%%%%%%%%%%%%%%%%%%%%%%%%%%%%%%%%%%
%%%%%%%%%%%%%%%%%%%%%%   END OF FRONT PAGE %%%%%%%%%%%%%%%%%%%%%%%%%%%%%%%%
%%%%%%%%%%%%%%%%%%%%%%%%%%%%%%%%%%%%%%%%%%%%%%%%%%%%%%%%%%%%%%%%%%%%%%%%%%%









